\pdfminorversion=4

% Modos:
% Definir el comando \handoutmode como "1" para compilar las diapositivas en
% modo handout
% al pdf generado ejecutarle el siguiente comando para poner varias diapo en una misma pagina:
%    pdfnup FILE.pdf --nup 2x3 --no-landscape --paper letterpaper --frame True
%
% Definir \handoutmode como algo distinto de "1" para compilar las diapositivas
% normalmente.
%
% Para configurar el modo desde afuera (la linea de comandos de latex) correr
% la compilacion como:
%
%   latex -output-directory=tmp -output-format=pdf '\def\handoutmode{1}\include{FILE.tex}'
%
\if\handoutmode1
    \documentclass[professionalfonts,handout]{beamer}
    \setbeameroption{show notes}
    \setbeamertemplate{note page}{\insertnote}
    \setbeamercolor{background canvas}{bg=white}
\else
    \documentclass[professionalfonts]{beamer}
\fi


% Template original the Beamer: Copyright 2004 by Till Tantau <tantau@users.sourceforge.net>.


\mode<presentation>
{
  \usetheme{metropolis}
  % or ...

  %\setbeamercovered{transparent} %hace que lo que esta covered se muestre transparente
}

\setbeamertemplate{navigation symbols}{}%remove navigation symbols

\usepackage[english]{babel}

\usepackage[latin1]{inputenc}

\usepackage{times}
\usepackage[T1]{fontenc}

% para poder escribir codigo fuente en las diapositivas
\usepackage{listings}

% para graficar diagramas simples y arboles de jerarquias
\usepackage{tikz}
\usepackage{tikz-qtree}
\usetikzlibrary{matrix,positioning}
\usetikzlibrary{shapes,arrows,chains,calc,decorations.pathmorphing}

\usepackage{xcolor}
\definecolor{darkgreen}{rgb}{0,.5,0}
\definecolor{darkblue}{rgb}{0,0,.7}
\lstdefinestyle{normal}{language=C++,       % lenguaje C++
   numbers=left,                            % enumerar las lineas
   keywordstyle=\color{darkblue}\textbf,    % color de las keywords
   stringstyle=\color{red},                 % color de los strings
   commentstyle=\color{darkgreen},          % color de los comentarios
   basicstyle=\color{black}\ttfamily\footnotesize\bfseries,     % color del texto en general
   morecomment=[l][\color{magenta}]{\#},    % coloreamos las intrucciones del precompilador (todo lo que empieza con #)
   ndkeywords={NULL,nullptr,siz,zer,mov,add},               % definimos una nuevas keywords como NULL y nullptr
   ndkeywordstyle=\color{violet},           % y las nuevas keywords tendran este color
   frame=simple,                            % simple, sin ningun marco o frame alrededor del codigo
   basewidth={0.55em,0.55em}                % tamano de las letras/lineas. usado para reducir el whitespace entre estas
}

\lstdefinestyle{normal33}{language=C++,       % lenguaje C++
   numbers=left,                            % enumerar las lineas
   keywordstyle=\color{darkblue}\textbf,    % color de las keywords
   stringstyle=\color{red},                 % color de los strings
   commentstyle=\color{darkgreen},          % color de los comentarios
   basicstyle=\color{black}\ttfamily\footnotesize\bfseries,     % color del texto en general
   morecomment=[l][\color{magenta}]{\#},    % coloreamos las intrucciones del precompilador (todo lo que empieza con #)
   ndkeywords={NULL,nullptr},               % definimos una nuevas keywords como NULL y nullptr
   ndkeywordstyle=\color{violet},           % y las nuevas keywords tendran este color
   frame=simple,                            % simple, sin ningun marco o frame alrededor del codigo
   basewidth={0.55em,0.55em}                % tamano de las letras/lineas. usado para reducir el whitespace entre estas
}
%mismo estilo, sin numeros
\lstdefinestyle{normalnonumbers}{language=C++,
   keywordstyle=\color{darkblue}\textbf,
   stringstyle=\color{red},           
   commentstyle=\color{darkgreen},    
   basicstyle=\color{black}\ttfamily\footnotesize\bfseries,
   morecomment=[l][\color{magenta}]{\#},
   ndkeywords={NULL,nullptr,move,add},          
   ndkeywordstyle=\color{violet}, 
   frame=simple,
   basewidth={0.55em,0.55em}
}


% mismo estilo pero todos los colores estan mas debiles como si se volvieran transparentes
% usado para poder resaltar el codigo.
\lstdefinestyle{dimmided}{language=C++,
   keywordstyle=\color{darkblue!30}\textbf,
   stringstyle=\color{red!30},
   commentstyle=\color{darkgreen!30},
   basicstyle=\color{black!30}\ttfamily\footnotesize\bfseries,
   morecomment=[l][\color{magenta!30}]{\#},
   ndkeywords={NULL,nullptr,siz,zer,mov,add},
   ndkeywordstyle=\color{violet!30}, 
   moredelim=**[is][\only<+->{\color{black}\lstset{style=normal}}]{@}{@}, % definimos que las lineas entre arrobas (@) tendran el estilo normal (con los colores a toda intensidad)
   frame=simple,
   basewidth={0.55em,0.55em}
}
\lstdefinestyle{dimmided42}{language=C++,
   keywordstyle=\color{darkblue!30}\textbf,
   stringstyle=\color{red!30},
   commentstyle=\color{darkgreen!30},
   basicstyle=\color{black!30}\ttfamily\footnotesize\bfseries,
   morecomment=[l][\color{magenta!30}]{\#},
   ndkeywords={NULL,nullptr,siz,zer,mov,add},
   ndkeywordstyle=\color{violet!30}, 
   moredelim=**[is][{\color{black}\lstset{style=normal}}]{@}{@}, % definimos que las lineas entre arrobas (@) tendran el estilo normal (con los colores a toda intensidad)
   frame=simple,
   basewidth={0.55em,0.55em}
}

\lstdefinelanguage{json}{
    literate=
     *{:}{{{\color{violet}{:}}}}{1}
      {,}{{{\color{violet}{,}}}}{1}
      {\{}{{{\color{darkgreen}{\{}}}}{1}
      {\}}{{{\color{darkgreen}{\}}}}}{1}
      {[}{{{\color{darkgreen}{[}}}}{1}
      {]}{{{\color{darkgreen}{]}}}}{1},
}


\lstdefinestyle{normaljson}{language=json,  % lenguaje json
   numbers=left,                            % enumerar las lineas
   keywordstyle=\color{darkblue}\textbf,    % color de las keywords
   stringstyle=\color{red},                 % color de los strings
   commentstyle=\color{red},                % color de los comentarios
   basicstyle=\color{black}\ttfamily\footnotesize\bfseries,     % color del texto en general
   morecomment=[l][\color{magenta}]{\#},    % coloreamos las intrucciones del precompilador (todo lo que empieza con #)
   ndkeywords={NULL,nullptr},               % definimos una nuevas keywords como NULL y nullptr
   ndkeywordstyle=\color{violet},           % y las nuevas keywords tendran este color
   frame=simple,                            % simple, sin ningun marco o frame alrededor del codigo
   basewidth={0.55em,0.55em}                % tamano de las letras/lineas. usado para reducir el whitespace entre estas
}

\lstdefinelanguage{http}{
    morekeywords={
        GET,
        PUT,
        POST,
        DELETE
    }
}

\lstdefinestyle{normalhttp}{language=http,  % lenguaje HTTP
   numbers=left,                            % enumerar las lineas
   keywordstyle=\color{darkblue}\textbf,    % color de las keywords
   stringstyle=\color{red},                 % color de los strings
   commentstyle=\color{red},                % color de los comentarios
   basicstyle=\color{black}\ttfamily\footnotesize\bfseries,     % color del texto en general
   morecomment=[l][\color{magenta}]{\#},    % coloreamos las intrucciones del precompilador (todo lo que empieza con #)
   ndkeywords={NULL,nullptr},               % definimos una nuevas keywords como NULL y nullptr
   ndkeywordstyle=\color{violet},           % y las nuevas keywords tendran este color
   frame=simple,                            % simple, sin ningun marco o frame alrededor del codigo
   basewidth={0.55em,0.55em}                % tamano de las letras/lineas. usado para reducir el whitespace entre estas
}

% Pone las constantes numericas con color (violeta en este caso):
% - los numeros que estan dentro de un string/comentarios NO son coloreados
% - los numeros por fuera que pertenecen a una palabra SON coloreados 
%   (buf1, por ejemplo, el "1" es coloreado cuando no lo deberias. UPS!!)
% - No incluye el signo
% 
%\lstset{literate=%
%  *{0}{{{\color{violet}0}}}1
%   {1}{{{\color{violet}1}}}1
%   {2}{{{\color{violet}2}}}1
%   {3}{{{\color{violet}3}}}1
%   {4}{{{\color{violet}4}}}1
%   {5}{{{\color{violet}5}}}1
%   {6}{{{\color{violet}6}}}1
%   {7}{{{\color{violet}7}}}1
%   {8}{{{\color{violet}8}}}1
%   {9}{{{\color{violet}9}}}1
%}

% Para resaltar lineas de codigo usando \btLstHLB{range} y \btLstHLB<overlay>{range}:
% Por ejemplo, 
%  \btLstHLB{3}       linea 3 resaltada
%  \btLstHLB{1-5}     lineas de la 1 a la 5 resaltadas
%  \btLstHLB<2>{3}    linea 3 resaltada solo en el slide 2
%
% \btLstHLB usa un color azul mientras que \btLstHLR usa un color rojo como fondo
\usepackage{lstlinebgrd}
\makeatletter
\newcount\bt@rangea
\newcount\bt@rangeb

\newcommand\btIfInRange[2]{%
   \global\let\bt@inrange\@secondoftwo%
   \edef\bt@rangelist{#2}%
   \foreach \range in \bt@rangelist {%
      \afterassignment\bt@getrangeb%
      \bt@rangea=0\range\relax%
      \pgfmathtruncatemacro\result{ ( #1 >= \bt@rangea) && (#1 <= \bt@rangeb) }%
      \ifnum\result=1\relax%
      \breakforeach%
      \global\let\bt@inrange\@firstoftwo%
      \fi%
   }%
   \bt@inrange%
}
\newcommand\bt@getrangeb{%
   \@ifnextchar\relax%
   {\bt@rangeb=\bt@rangea}%
   {\@getrangeb}%
}
\def\@getrangeb-#1\relax{%
   \ifx\relax#1\relax%
      \bt@rangeb=100000%   \maxdimen is too large for pgfmath
   \else%
      \bt@rangeb=#1\relax%
   \fi%
}

%%%%%%%%%%%%%%%%%%%%%%%%%%%%%%%%%%%%%%%%%%%%%%%%%%%%%%%%%%%%%%%%%%%%%%%%%%%%%%
%
% \btLstHL<overlay spec>{range list}
%
\newcommand<>{\btLstHLB}[1]{%
   \only#2{\btIfInRange{\value{lstnumber}}{#1}{\color{blue!30}}% blue
   {\def\lst@linebgrdcmd####1####2####3{}}}%
}%
\newcommand<>{\btLstHLR}[1]{%
   \only#2{\btIfInRange{\value{lstnumber}}{#1}{\color{red!30}}% red
   {\def\lst@linebgrdcmd####1####2####3{}}}%
}%
%
%
%%%%%%%%%%%%%%%%%%%%%%

\makeatother


% sin fecha
\date{}

\author[7542]{Di Paola Mart\'in \\ \texttt{martinp.dipaola <at> gmail.com} }

\institute[Universidad de Buenos Aires]
{
   Facultad de Ingenier\'ia\\
   Universidad de Buenos Aires
}

% Definimos una imagen para que este en cada slide
%\pgfdeclareimage[height=0.5cm]{university-logo}{imgs/fiuba.png}
%\logo{\pgfuseimage{university-logo}}

%% Definir estas en el documento final
%%
%% \title%
%% {Programaci\'on gen\'erica y templates en C++}
%%
%% \subject{Programaci\'on gen\'erica y templates en C++}


%%%%%
% At begin of each Section do nothing; at each Subsection show the Section and Subsection names
% If the Section doesn't have a Subsection, then YOU must to add a unnumbered Subsection like \subsection*{}
% so the AtBeginSubsection will get triggered
\AtBeginSection{}
\AtBeginSubsection[\frame{\subsectionpage}]{\frame{\subsectionpage}}
%
%%%%%




\title%
{Servidores multi-cliente (draft)}

\subject{Servidores multi-cliente (draft)}

\begin{document}

% Keys to support piece-wise uncovering of elements in TikZ pictures:
% \node[visible on=<2->](foo){Foo}
% \node[visible on=<{2,4}>](bar){Bar}   % put braces around comma expressions
%
% Internally works by setting opacity=0 when invisible, which has the
% adavantage (compared to \node<2->(foo){Foo} that the node is always there, hence
% always consumes space that (foo) is always available.
%
% The actual command that implements the invisibility can be overriden
% by altering the style invisible. For instance \tikzsset{invisible/.style={opacity=0.2}}
% would dim the "invisible" parts. Alternatively, the color might be set to white, if the
% output driver does not support transparencies (e.g., PS)
%
\tikzset{
 nodeinvisible/.style={opacity=.4,fill=gray},
 nodevisible on/.style={alt={#1{}{nodeinvisible}}},
 arrowinvisible/.style={opacity=.4},
 arrowvisible on/.style={alt={#1{}{arrowinvisible}}},
 alt/.code args={<#1>#2#3}{%
   \alt<#1>{\pgfkeysalso{#2}}{\pgfkeysalso{#3}} % \pgfkeysalso doesn't change the path
 },
}

\begin{frame}
   \titlepage
\end{frame}


\section{Servidores multi-cliente (draft)}
\subsection{}
\begin{frame}[fragile,label=M1]{Simple: servidor para un solo cliente a la vez}{}

% -------------------------------------------------
% Start the picture
\begin{tikzpicture}[%
    >=latex,              % Nice arrows; your taste may be different
    start chain=going below,    % General flow is top-to-bottom
    node distance=4mm and 60mm, % Global setup of box spacing
    every join/.style={norm},   % Default linetype for connecting boxes
    ]\footnotesize
% -------------------------------------------------
% A few box styles
% <on chain> *and* <on grid> reduce the need for manual relative
% positioning of nodes
\tikzset{
% Parece que esto funciona como un sistema de herencia (proc hereda de base; test hereda
% de base tambien; term hereda de proc...)
% Todo esto son definiciones de estilos
  base/.style={draw, on chain, on grid, align=left, minimum height=4ex},
  proc/.style={base, rounded corners},
  % coord node style is used for placing corners of connecting lines
  coord/.style={coordinate, on chain, on grid, node distance=6mm and 25mm},
  namethread/.style={base, align=center, draw=none, node distance=6mm and 25mm},
  % -------------------------------------------------
  % Connector line styles for different parts of the diagram
  norm/.style={->, draw, black, line width=0.03cm},
  ret/.style={->, draw, black, line width=0.03cm},
  snakearrow/.style={-, decorate, line width=0.03cm, decoration={snake,amplitude=.6mm,segment length=2mm,post length=1mm}}
}
% -------------------------------------------------
% Start by placing the nodes
% Al parecer se pone todos los styles entre [], un nombre o id entre () y el
% texto entre {}
\node [proc, nodevisible on={<1,2>}] (sktaccept) {
   \lstinline[style=normal]!peer = skt.accept()!
};

\node [proc, nodevisible on={<1,2>}] (peerreadwrite) {
   \lstinline[style=normal]!peer.send/recv!
};
\node [proc, nodevisible on={<1,2>}] (peershutclose) {
   \lstinline[style=normal]!peer.shutdown(RDWR)!\\
   \lstinline[style=normal]!peer.close()!
};

\node [proc, nodevisible on={<1,3>}] (sktclose) {
   \lstinline[style=normal]!skt.close()!
};

% Now we place the coordinate nodes for the connectors with angles, or
% with annotations. We also mark them for debugging.
\node [coord, left=of peershutclose] (c1)  {};

\node [namethread, above=4em of sktaccept] (c0)  {(main)};
\node [coord, below=3em of sktclose] (c3)  {};


\draw [->, norm, arrowvisible on={<1,2>}] (sktaccept) -- (peerreadwrite);
\draw [->, norm, arrowvisible on={<1,2>}] (peerreadwrite) -- (peershutclose);

\draw [->, ret, arrowvisible on={<1,2>}] (peershutclose.west) to[out=180,in=180] (sktaccept.west);
\draw [->, ret, arrowvisible on={<1,3>}] (sktaccept.east) to[out=0,in=0] (sktclose);

\draw [snakearrow] (c0) -- (sktaccept);
\draw [snakearrow, arrowvisible on={<1,3>}] (sktclose) -- (c3);

\end{tikzpicture}

% =================================================

\end{frame}
\note[itemize] {
\item El programa es un bucle simple, se espera una conexi\'on remota, se la procesa y se repite el proceso
\item Solo cuando \lstinline[style=normal]!accept! falla por recibir una se\~nal, el bucle finaliza.
\item Este esquema solo soporta un cliente a la vez y no permite hacer cosas en paralelo mientras se espera a un nuevo cliente ni mientras se habla con uno ya conectado.
\item Es muy simple. Usado en servidores RPC y servidores Web simples (o dummies).
}

\begin{frame}[fragile,label=M2]{Servidor multi-cliente (draft)}{}

% -------------------------------------------------
% Start the picture
\begin{tikzpicture}[%
    >=latex,              % Nice arrows; your taste may be different
    start chain=going below,    % General flow is top-to-bottom
    node distance=4mm and 50mm, % Global setup of box spacing
    every join/.style={norm},   % Default linetype for connecting boxes
    ]\footnotesize

\tikzset{
  base/.style={draw, on chain, on grid, align=left, minimum height=4ex},
  proc/.style={base, rounded corners},
  % coord node style is used for placing corners of connecting lines
  coord/.style={coordinate, on chain, on grid, node distance=6mm and 25mm},
  namethread/.style={base, align=center, draw=none, node distance=6mm and 25mm},
  % -------------------------------------------------
  % Connector line styles for different parts of the diagram
  norm/.style={->, draw, black, line width=0.03cm},
  ret/.style={->, draw, black, line width=0.03cm},
  snakearrow/.style={-, decorate, line width=0.03cm, decoration={snake,amplitude=.6mm,segment length=2mm,post length=1mm}},
  threadarrow/.style={->, dashed, draw, black, line width=0.03cm},
}


% Hilo main
\node [proc, nodevisible on={<1,2,3>}] (sktaccept) {
   \lstinline[style=normal]!peer = skt.accept()!\\
    \lstinline[style=normal]!clients.add(!\\
     \lstinline[style=normal]!    ThClient(peer))!
};

\node [proc, nodevisible on={<1,2,3>}] (reap) {
   \lstinline[style=normal]!for (cli in clients)!\\
    \lstinline[style=normal]!   if cli.is_dead()!\\
   \lstinline[style=normal]!      cli.join()!\\
   \lstinline[style=normal]!      free/delete cli!
};

\node [proc, nodevisible on={<1,4>}] (stopreap) {
    \lstinline[style=normal]!for (cli in clients)!\\
   \lstinline[style=normal]!   cli.stop()!\\
   \lstinline[style=normal]!   cli.join()!\\
   \lstinline[style=normal]!   free/delete cli!
};


\node [proc, nodevisible on={<1,4>}] (sktclose) {
   \lstinline[style=normal]!skt.close()!
};
% hilo cliente
\node [proc, right=4cm of reap, nodevisible on={<1,2>}] (peerreadwrite) {
   \lstinline[style=normal]!peer.send/recv!
};


\node [namethread, above=4em of sktaccept] (abovesktaccept)  {(main)};
\node [namethread, above=4em of peerreadwrite] (abovepeerreadwrite)  {(cliente i)};

\node [coord, below=3em of sktclose] (belowsktclose)  {};
\node [coord, below=3em of peerreadwrite] (belowpeerreadwrite)  {};

\draw [->, norm,arrowvisible on={<1,2,3>}] (sktaccept) -- (reap);
\draw [->, norm,arrowvisible on={<1,4>}] (stopreap) -- (sktclose);

\draw [->, ret,arrowvisible on={<1,2,3>}] (reap.west) to[out=115,in=155,distance=6em] (sktaccept.north);
\draw [->, ret,arrowvisible on={<1,2>}] (peerreadwrite.west) to[out=160,in=115,distance=2.3em] (peerreadwrite.north);

\draw [->, ret,arrowvisible on={<1,4>}] (sktaccept.east) to[out=280,in=5,distance=8em] (stopreap.north);

\path (sktaccept.east) to node [near start, yshift=0.8em, xshift=4em] {\lstinline[style=normal]!!} (sktaccept);
  \draw [->,threadarrow,arrowvisible on={<1,2>}] (sktaccept.east) -- (abovepeerreadwrite.south);

\draw [->,threadarrow,arrowvisible on={<1,3>}] (belowpeerreadwrite.south) -- (reap.east);
\draw [->,threadarrow,arrowvisible on={<1,4>}] (belowpeerreadwrite.south) -- (stopreap.east);

\draw [snakearrow] (abovesktaccept) -- (sktaccept);
\draw [snakearrow,arrowvisible on={<1,2>}] (abovepeerreadwrite) -- (peerreadwrite);

\draw [snakearrow, arrowvisible on={<1,4>}] (sktclose) -- (belowsktclose);
\draw [snakearrow, arrowvisible on={<1,3,4>}] (peerreadwrite) -- (belowpeerreadwrite);

\end{tikzpicture}

% =================================================

\end{frame}

\begin{frame}[fragile,label=M4]{Frenar un hilo}{}
   \begin{columns}
      \begin{column}{.55\linewidth}
         \begin{lstlisting}[style=normal]
class ThClient:public Thread {
   std::atomic<bool> keep_talking;
   std::atomic<bool> is_running;
   Socket peer;

   public:
   ThClient():keep_talking(true),
              is_running(true) {}

   virtual void run() {
      while (keep_talking) {
         ...
         peer.send(...);
         ...
         peer.recv(...);
         ...
      }
      is_running = false;
   }
};
         \end{lstlisting}
      \end{column}
      \begin{column}{.45\linewidth}
         \begin{lstlisting}[style=normal]
// Violento pero efectivo
void ThClient::stop() {
   keep_talking = false;
   peer.shutdown();
   peer.close();
};
         \end{lstlisting}

         \begin{lstlisting}[style=normal]
// Polite pero peligroso
void ThClient::stop() {
   keep_talking = false;
};
         \end{lstlisting}
      \end{column}
   \end{columns}
\end{frame}
\note[itemize] {
\item Como frenar un hilo? La librer\'ia \lstinline[style=normal]!pthread! ofrece una manera gen\'erica de frenar o matar a un hilo (stop/kill) pero deja \alert{los recursos sin finalizar}. \alert{No usar}.
\item Frenar un hilo correctamente depender\'a de la naturaleza del hilo (depende de la aplicaci\'on en cuesti\'on, no hay una soluci\'on general).
\item Hay dos variantes posibles: forzar un cierre o decirle al hilo que cuando pueda \'el mismo finalize.
\item Si el hilo esta bloqueado en una operaci\'ion de sockets, se puede hacer un \lstinline[style=normal]!shutdown/close! del socket para forzar un cierre. Obviamente si el hilo estaba en el medio de una comunicaci\'on, el trabajo puedo quedar trunco o corrupto.
\item Usar un \lstinline[style=normal]!bool! para decirle al hilo que finalize. El hilo puede terminar su conversaci\'on y cerrar ordenamdamente: es m\'as seguro pero si el hilo est\'a bloqueado \alert{no} se desbloquear\'a.
\item Y si el hilo esta bloqueado en otra operaci\'on? Como en un \lstinline[style=normal]!queue_safe.pull()!? No hay una soluci\'on gen\'erica.
}

\begin{frame}[fragile,label=M3]{Servidor multi-cliente (final): cierre ordenado}{}

% -------------------------------------------------
% Start the picture
\begin{tikzpicture}[%
    >=latex,              % Nice arrows; your taste may be different
    start chain=going below,    % General flow is top-to-bottom
    node distance=4mm and 50mm, % Global setup of box spacing
    every join/.style={norm},   % Default linetype for connecting boxes
    ]\footnotesize

\tikzset{
  base/.style={draw, on chain, on grid, align=left, minimum height=4ex},
  proc/.style={base, rounded corners},
  % coord node style is used for placing corners of connecting lines
  coord/.style={coordinate, on chain, on grid, node distance=6mm and 25mm},
  namethread/.style={base, align=center, draw=none, node distance=6mm and 25mm},
  % -------------------------------------------------
  % Connector line styles for different parts of the diagram
  norm/.style={->, draw, black, line width=0.03cm},
  ret/.style={->, draw, black, line width=0.03cm},
  snakearrow/.style={-, decorate, line width=0.03cm, decoration={snake,amplitude=.6mm,segment length=2mm,post length=1mm}},
  threadarrow/.style={->, dashed, draw, black, line width=0.03cm},
}

%Primer hilo (main)
\node [coord] (mainfork)  {};

\node [proc, below=3em of mainfork, nodevisible on={<1,2>}] (leerstdin) {
   \lstinline[style=normal]!c = cin.getc()!
};

\node [proc, below=14em of leerstdin, nodevisible on={<1,3,4>}] (sktclose) {
   \lstinline[style=normal]!skt.close()!
};

\node [proc, nodevisible on={<1,4>}] (thjoin) {
   \lstinline[style=normal]!aceptador.join()!
};

% Hilo aceptador
\node [proc, right=3.8cm of leerstdin, nodevisible on={<1,2>}] (sktaccept) {
   \lstinline[style=normal]!peer = skt.accept()!\\
    \lstinline[style=normal]!clients.add(!\\
     \lstinline[style=normal]!    ThClient(peer))!
};

\node [proc, nodevisible on={<1,2>}] (reap) {
   \lstinline[style=normal]!for (cli in clients)!\\
    \lstinline[style=normal]!   if cli.is_dead()!\\
   \lstinline[style=normal]!      cli.join()!\\
   \lstinline[style=normal]!      free/delete cli!
};

\node [proc, nodevisible on={<1,3>}] (stopreap) {
    \lstinline[style=normal]!for (cli in clients)!\\
   \lstinline[style=normal]!   cli.stop()!\\
   \lstinline[style=normal]!   cli.join()!\\
   \lstinline[style=normal]!   free/delete cli!
};

% hilo cliente
\node [proc, right=4cm of reap, nodevisible on={<1,2,3>}] (peerreadwrite) {
   \lstinline[style=normal]!peer.send/recv!
};


\node [namethread, above=1em of mainfork] (abovemainfork)  {(main)};
\node [namethread, above=4em of sktaccept] (abovesktaccept)  {(aceptador)};
\node [namethread, above=4em of peerreadwrite] (abovepeerreadwrite)  {(cliente i)};

\node [coord, below=3em of thjoin] (belowthjoin)  {};
\node [coord, below=4em of stopreap] (belowstopreap)  {};
\node [coord, below=3em of peerreadwrite] (belowpeerreadwrite)  {};

\draw [->, norm,arrowvisible on={<1,3>}] (leerstdin) -- (sktclose);
\draw [->, norm,arrowvisible on={<1,4>}] (sktclose) -- (thjoin);
\draw [->, norm,arrowvisible on={<1,2>}] (sktaccept) -- (reap);

\draw [->, ret,arrowvisible on={<1,2>}] (leerstdin.west) to[out=160,in=115,distance=2.3em] (leerstdin.north);
\draw [->, ret,arrowvisible on={<1,2>}] (reap.west) to[out=115,in=155,distance=6em] (sktaccept.north);
\draw [->, ret,arrowvisible on={<1,2,3>}] (peerreadwrite.west) to[out=160,in=115,distance=2.3em] (peerreadwrite.north);

\draw [->, ret,arrowvisible on={<1,3>}] (sktaccept.east) to[out=280,in=5,distance=8em] (stopreap.north);

\draw [->,threadarrow,arrowvisible on={<1,2>}] (mainfork.east) -- (abovesktaccept.south);
\draw [->,threadarrow,arrowvisible on={<1,2>}] (sktaccept.east) -- (abovepeerreadwrite.south);

\draw [->,threadarrow,arrowvisible on={<1,2>}] (belowpeerreadwrite.south) -- (reap.east);
\draw [->,threadarrow,arrowvisible on={<1,3,4>}] (belowpeerreadwrite.south) -- (stopreap.east);
\draw [->,threadarrow,arrowvisible on={<1,4>}] (belowstopreap.south) -- (thjoin.east);

\draw [snakearrow] (abovemainfork) -- (mainfork);
\draw [snakearrow] (mainfork) -- (leerstdin);
\draw [snakearrow] (abovesktaccept) -- (sktaccept);
\draw [snakearrow] (abovepeerreadwrite) -- (peerreadwrite);

\draw [snakearrow] (thjoin) -- (belowthjoin);
\draw [snakearrow] (stopreap) -- (belowstopreap);
\draw [snakearrow] (peerreadwrite) -- (belowpeerreadwrite);

\end{tikzpicture}
\end{frame}

\begin{frame}[fragile,label=M5]{Hilos de comunicaci\'on}{}

% -------------------------------------------------
% Start the picture
\begin{tikzpicture}[%
    >=latex,              % Nice arrows; your taste may be different
    start chain=going below,    % General flow is top-to-bottom
    node distance=4mm and 50mm, % Global setup of box spacing
    every join/.style={norm},   % Default linetype for connecting boxes
    ]\footnotesize

\tikzset{
  base/.style={draw, on chain, on grid, align=left, minimum height=4ex},
  proc/.style={base, rounded corners},
  % coord node style is used for placing corners of connecting lines
  coord/.style={coordinate, on chain, on grid, node distance=6mm and 25mm},
  namethread/.style={base, align=center, draw=none, node distance=6mm and 25mm},
  % -------------------------------------------------
  % Connector line styles for different parts of the diagram
  norm/.style={->, draw, black, line width=0.03cm},
  ret/.style={->, draw, black, line width=0.03cm},
  snakearrow/.style={-, decorate, line width=0.03cm, decoration={snake,amplitude=.6mm,segment length=2mm,post length=1mm}},
  threadarrow/.style={->, dashed, draw, black, line width=0.03cm},
}

\node [proc] (peerreadwrite) {
   \lstinline[style=normal]!peer.send/recv!
};

\node [proc, right=5cm of peerreadwrite] (peerwrite) {
   \lstinline[style=normal]!peer.send!
};
\node [proc, right=2.6cm of peerwrite] (peerread) {
   \lstinline[style=normal]!peer.recv!
};

\node [namethread, above=4em of peerreadwrite] (abovepeerreadwrite)  {(cliente i)};
\node [namethread, above=4em of peerwrite] (abovepeerwrite)  {(cliente i, escritor)};
\node [namethread, above=4em of peerread] (abovepeerread)  {(cliente i, lector)};

\node [coord, below=3em of peerreadwrite] (belowpeerreadwrite)  {};
\node [coord, below=3em of peerwrite] (belowpeerwrite)  {};
\node [coord, below=3em of peerread] (belowpeerread)  {};


\draw [snakearrow] (abovepeerreadwrite) -- (peerreadwrite);
\draw [snakearrow] (abovepeerwrite) -- (peerwrite);
\draw [snakearrow] (abovepeerread) -- (peerread);

\draw [snakearrow] (peerreadwrite) -- (belowpeerreadwrite);
\draw [snakearrow] (peerwrite) -- (belowpeerwrite);
\draw [snakearrow] (peerread) -- (belowpeerread);

\end{tikzpicture}
\end{frame}
\note[itemize] {
\item Un socket puede ser leido (\lstinline[style=normal]!recv!) por un hilo y escrito (\lstinline[style=normal]!send!) por otro. La lectura no entra en conflicto con la escritura.
\item Pero hacer un \lstinline[style=normal]!send! (o un \lstinline[style=normal]!recv!) sobre un mismo socket desde m\'ultiples hilos trae problemas de concurrencia!
}

\end{document}


