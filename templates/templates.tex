\pdfminorversion=4

% Modos:
% Definir el comando \handoutmode como "1" para compilar las diapositivas en
% modo handout
% al pdf generado ejecutarle el siguiente comando para poner varias diapo en una misma pagina:
%    pdfnup FILE.pdf --nup 2x3 --no-landscape --paper letterpaper --frame True
%
% Definir \handoutmode como algo distinto de "1" para compilar las diapositivas
% normalmente.
%
% Para configurar el modo desde afuera (la linea de comandos de latex) correr
% la compilacion como:
%
%   latex -output-directory=tmp -output-format=pdf '\def\handoutmode{1}\include{FILE.tex}'
%
\if\handoutmode1
    \documentclass[professionalfonts,handout]{beamer}
    \setbeameroption{show notes}
    \setbeamertemplate{note page}{\insertnote}
    \setbeamercolor{background canvas}{bg=white}
\else
    \documentclass[professionalfonts]{beamer}
\fi


% Template original the Beamer: Copyright 2004 by Till Tantau <tantau@users.sourceforge.net>.


\mode<presentation>
{
  \usetheme{metropolis}
  % or ...

  %\setbeamercovered{transparent} %hace que lo que esta covered se muestre transparente
}

\setbeamertemplate{navigation symbols}{}%remove navigation symbols

\usepackage[english]{babel}

\usepackage[latin1]{inputenc}

\usepackage{times}
\usepackage[T1]{fontenc}

% para poder escribir codigo fuente en las diapositivas
\usepackage{listings}

% para graficar diagramas simples y arboles de jerarquias
\usepackage{tikz}
\usepackage{tikz-qtree}
\usetikzlibrary{matrix,positioning}
\usetikzlibrary{shapes,arrows,chains,calc,decorations.pathmorphing}

\usepackage{xcolor}
\definecolor{darkgreen}{rgb}{0,.5,0}
\definecolor{darkblue}{rgb}{0,0,.7}
\lstdefinestyle{normal}{language=C++,       % lenguaje C++
   numbers=left,                            % enumerar las lineas
   keywordstyle=\color{darkblue}\textbf,    % color de las keywords
   stringstyle=\color{red},                 % color de los strings
   commentstyle=\color{darkgreen},          % color de los comentarios
   basicstyle=\color{black}\ttfamily\footnotesize\bfseries,     % color del texto en general
   morecomment=[l][\color{magenta}]{\#},    % coloreamos las intrucciones del precompilador (todo lo que empieza con #)
   ndkeywords={NULL,nullptr,siz,zer,mov,add},               % definimos una nuevas keywords como NULL y nullptr
   ndkeywordstyle=\color{violet},           % y las nuevas keywords tendran este color
   frame=simple,                            % simple, sin ningun marco o frame alrededor del codigo
   basewidth={0.55em,0.55em}                % tamano de las letras/lineas. usado para reducir el whitespace entre estas
}

\lstdefinestyle{normal33}{language=C++,       % lenguaje C++
   numbers=left,                            % enumerar las lineas
   keywordstyle=\color{darkblue}\textbf,    % color de las keywords
   stringstyle=\color{red},                 % color de los strings
   commentstyle=\color{darkgreen},          % color de los comentarios
   basicstyle=\color{black}\ttfamily\footnotesize\bfseries,     % color del texto en general
   morecomment=[l][\color{magenta}]{\#},    % coloreamos las intrucciones del precompilador (todo lo que empieza con #)
   ndkeywords={NULL,nullptr},               % definimos una nuevas keywords como NULL y nullptr
   ndkeywordstyle=\color{violet},           % y las nuevas keywords tendran este color
   frame=simple,                            % simple, sin ningun marco o frame alrededor del codigo
   basewidth={0.55em,0.55em}                % tamano de las letras/lineas. usado para reducir el whitespace entre estas
}
%mismo estilo, sin numeros
\lstdefinestyle{normalnonumbers}{language=C++,
   keywordstyle=\color{darkblue}\textbf,
   stringstyle=\color{red},           
   commentstyle=\color{darkgreen},    
   basicstyle=\color{black}\ttfamily\footnotesize\bfseries,
   morecomment=[l][\color{magenta}]{\#},
   ndkeywords={NULL,nullptr,move,add},          
   ndkeywordstyle=\color{violet}, 
   frame=simple,
   basewidth={0.55em,0.55em}
}


% mismo estilo pero todos los colores estan mas debiles como si se volvieran transparentes
% usado para poder resaltar el codigo.
\lstdefinestyle{dimmided}{language=C++,
   keywordstyle=\color{darkblue!30}\textbf,
   stringstyle=\color{red!30},
   commentstyle=\color{darkgreen!30},
   basicstyle=\color{black!30}\ttfamily\footnotesize\bfseries,
   morecomment=[l][\color{magenta!30}]{\#},
   ndkeywords={NULL,nullptr,siz,zer,mov,add},
   ndkeywordstyle=\color{violet!30}, 
   moredelim=**[is][\only<+->{\color{black}\lstset{style=normal}}]{@}{@}, % definimos que las lineas entre arrobas (@) tendran el estilo normal (con los colores a toda intensidad)
   frame=simple,
   basewidth={0.55em,0.55em}
}
\lstdefinestyle{dimmided42}{language=C++,
   keywordstyle=\color{darkblue!30}\textbf,
   stringstyle=\color{red!30},
   commentstyle=\color{darkgreen!30},
   basicstyle=\color{black!30}\ttfamily\footnotesize\bfseries,
   morecomment=[l][\color{magenta!30}]{\#},
   ndkeywords={NULL,nullptr,siz,zer,mov,add},
   ndkeywordstyle=\color{violet!30}, 
   moredelim=**[is][{\color{black}\lstset{style=normal}}]{@}{@}, % definimos que las lineas entre arrobas (@) tendran el estilo normal (con los colores a toda intensidad)
   frame=simple,
   basewidth={0.55em,0.55em}
}

\lstdefinelanguage{json}{
    literate=
     *{:}{{{\color{violet}{:}}}}{1}
      {,}{{{\color{violet}{,}}}}{1}
      {\{}{{{\color{darkgreen}{\{}}}}{1}
      {\}}{{{\color{darkgreen}{\}}}}}{1}
      {[}{{{\color{darkgreen}{[}}}}{1}
      {]}{{{\color{darkgreen}{]}}}}{1},
}


\lstdefinestyle{normaljson}{language=json,  % lenguaje json
   numbers=left,                            % enumerar las lineas
   keywordstyle=\color{darkblue}\textbf,    % color de las keywords
   stringstyle=\color{red},                 % color de los strings
   commentstyle=\color{red},                % color de los comentarios
   basicstyle=\color{black}\ttfamily\footnotesize\bfseries,     % color del texto en general
   morecomment=[l][\color{magenta}]{\#},    % coloreamos las intrucciones del precompilador (todo lo que empieza con #)
   ndkeywords={NULL,nullptr},               % definimos una nuevas keywords como NULL y nullptr
   ndkeywordstyle=\color{violet},           % y las nuevas keywords tendran este color
   frame=simple,                            % simple, sin ningun marco o frame alrededor del codigo
   basewidth={0.55em,0.55em}                % tamano de las letras/lineas. usado para reducir el whitespace entre estas
}

\lstdefinelanguage{http}{
    morekeywords={
        GET,
        PUT,
        POST,
        DELETE
    }
}

\lstdefinestyle{normalhttp}{language=http,  % lenguaje HTTP
   numbers=left,                            % enumerar las lineas
   keywordstyle=\color{darkblue}\textbf,    % color de las keywords
   stringstyle=\color{red},                 % color de los strings
   commentstyle=\color{red},                % color de los comentarios
   basicstyle=\color{black}\ttfamily\footnotesize\bfseries,     % color del texto en general
   morecomment=[l][\color{magenta}]{\#},    % coloreamos las intrucciones del precompilador (todo lo que empieza con #)
   ndkeywords={NULL,nullptr},               % definimos una nuevas keywords como NULL y nullptr
   ndkeywordstyle=\color{violet},           % y las nuevas keywords tendran este color
   frame=simple,                            % simple, sin ningun marco o frame alrededor del codigo
   basewidth={0.55em,0.55em}                % tamano de las letras/lineas. usado para reducir el whitespace entre estas
}

% Pone las constantes numericas con color (violeta en este caso):
% - los numeros que estan dentro de un string/comentarios NO son coloreados
% - los numeros por fuera que pertenecen a una palabra SON coloreados 
%   (buf1, por ejemplo, el "1" es coloreado cuando no lo deberias. UPS!!)
% - No incluye el signo
% 
%\lstset{literate=%
%  *{0}{{{\color{violet}0}}}1
%   {1}{{{\color{violet}1}}}1
%   {2}{{{\color{violet}2}}}1
%   {3}{{{\color{violet}3}}}1
%   {4}{{{\color{violet}4}}}1
%   {5}{{{\color{violet}5}}}1
%   {6}{{{\color{violet}6}}}1
%   {7}{{{\color{violet}7}}}1
%   {8}{{{\color{violet}8}}}1
%   {9}{{{\color{violet}9}}}1
%}

% Para resaltar lineas de codigo usando \btLstHLB{range} y \btLstHLB<overlay>{range}:
% Por ejemplo, 
%  \btLstHLB{3}       linea 3 resaltada
%  \btLstHLB{1-5}     lineas de la 1 a la 5 resaltadas
%  \btLstHLB<2>{3}    linea 3 resaltada solo en el slide 2
%
% \btLstHLB usa un color azul mientras que \btLstHLR usa un color rojo como fondo
\usepackage{lstlinebgrd}
\makeatletter
\newcount\bt@rangea
\newcount\bt@rangeb

\newcommand\btIfInRange[2]{%
   \global\let\bt@inrange\@secondoftwo%
   \edef\bt@rangelist{#2}%
   \foreach \range in \bt@rangelist {%
      \afterassignment\bt@getrangeb%
      \bt@rangea=0\range\relax%
      \pgfmathtruncatemacro\result{ ( #1 >= \bt@rangea) && (#1 <= \bt@rangeb) }%
      \ifnum\result=1\relax%
      \breakforeach%
      \global\let\bt@inrange\@firstoftwo%
      \fi%
   }%
   \bt@inrange%
}
\newcommand\bt@getrangeb{%
   \@ifnextchar\relax%
   {\bt@rangeb=\bt@rangea}%
   {\@getrangeb}%
}
\def\@getrangeb-#1\relax{%
   \ifx\relax#1\relax%
      \bt@rangeb=100000%   \maxdimen is too large for pgfmath
   \else%
      \bt@rangeb=#1\relax%
   \fi%
}

%%%%%%%%%%%%%%%%%%%%%%%%%%%%%%%%%%%%%%%%%%%%%%%%%%%%%%%%%%%%%%%%%%%%%%%%%%%%%%
%
% \btLstHL<overlay spec>{range list}
%
\newcommand<>{\btLstHLB}[1]{%
   \only#2{\btIfInRange{\value{lstnumber}}{#1}{\color{blue!30}}% blue
   {\def\lst@linebgrdcmd####1####2####3{}}}%
}%
\newcommand<>{\btLstHLR}[1]{%
   \only#2{\btIfInRange{\value{lstnumber}}{#1}{\color{red!30}}% red
   {\def\lst@linebgrdcmd####1####2####3{}}}%
}%
%
%
%%%%%%%%%%%%%%%%%%%%%%

\makeatother


% sin fecha
\date{}

\author[7542]{Di Paola Mart\'in \\ \texttt{martinp.dipaola <at> gmail.com} }

\institute[Universidad de Buenos Aires]
{
   Facultad de Ingenier\'ia\\
   Universidad de Buenos Aires
}

% Definimos una imagen para que este en cada slide
%\pgfdeclareimage[height=0.5cm]{university-logo}{imgs/fiuba.png}
%\logo{\pgfuseimage{university-logo}}

%% Definir estas en el documento final
%%
%% \title%
%% {Programaci\'on gen\'erica y templates en C++}
%%
%% \subject{Programaci\'on gen\'erica y templates en C++}


%%%%%
% At begin of each Section do nothing; at each Subsection show the Section and Subsection names
% If the Section doesn't have a Subsection, then YOU must to add a unnumbered Subsection like \subsection*{}
% so the AtBeginSubsection will get triggered
\AtBeginSection{}
\AtBeginSubsection[\frame{\subsectionpage}]{\frame{\subsectionpage}}
%
%%%%%




\title%
{Programaci\'on gen\'erica y templates en C++}

\subject{Programaci\'on gen\'erica y templates en C++}

\begin{document}

\begin{frame}
   \titlepage
\end{frame}

\begin{frame}{De qu\'e va esto?}
   \tableofcontents
\end{frame}


\section{Programaci\'on gen\'erica}
\subsection{Motivaci\'on}

\begin{frame}[fragile]{Juego de buscar diferencias}
   \begin{columns}
      \begin{column}{.5\linewidth}
         \begin{lstlisting}[style=normal,linebackgroundcolor={%
         \only<1>{\def\lst@linebgrdcmd####1####2####3{}}%
         \btLstHLB<2>{2}% espacio
         \btLstHLB<3>{6}% operador asignacion
         \btLstHLB<4>{10}% constructor por copia
   }]
class Array_int {
   int data[64];

   public:
   void set(int p, int v){
      data[p] = v;
   }

   int get(int p) {
      return data[p];
   }
};
         \end{lstlisting}
      \end{column}
      \begin{column}{.45\linewidth}
         \begin{lstlisting}[style=normal,linebackgroundcolor={%
         \only<1>{\def\lst@linebgrdcmd####1####2####3{}}%
         \btLstHLB<2>{2}% espacio
         \btLstHLB<3>{6}% operador asignacion
         \btLstHLB<4>{10}% constructor por copia
   }]
class Array_char {
   char data[64];

   public:
   void set(int p, char v){
      data[p] = v;
   }

   char get(int p) {
      return data[p];
   }
};
         \end{lstlisting}
      \end{column}
   \end{columns}
\only<2>{Reserva de espacio distintos}
\only<3>{Invocaci\'on de c\'odigo distintos: operador asignaci\'on}
\only<4>{Operador copia tambi\'en (y hay otros m\'as...)}
\end{frame}
\note[itemize] {
\item Imaginemos que necesitamos un array de \lstinline[style=normal]!64 int!s asi como tambi\'en de \lstinline[style=normal]!64 char!s. De las dos implementaciones, que diferencias hay?
\item Aunque no lo parezca, hay diferencias importantes desde el punto de vista del compilador y del c\'odigo m\'aquina generado.
\item Primero, reservan espacios distintos: \lstinline[style=normal]!64*sizeof(int)! contra \lstinline[style=normal]!64*sizeof(char)!
\item Segundo, invocan a c\'odigo (operadores) distintos: por ejemplo el operador asignaci\'on
\item Otros c\'odigos tambi\'en: el constructor por copia, posiblemente el constructor por default y el destructor.
\item Y todas estas diferencias por tan solo debido al cambio del tipo \lstinline[style=normal]!int! por \lstinline[style=normal]!char!
}


\begin{frame}[plain,fragile]{Alternariva I: void*}
   \begin{columns}[t]
      \begin{column}{.55\linewidth}
         \begin{lstlisting}[style=normal,linebackgroundcolor={%
               \only<1>{\def\lst@linebgrdcmd####1####2####3{}}%
               \btLstHLB<2>{7-8}% memcpy, copia bit a bit
               \btLstHLB<3>{12}% casteo!
         }]
class Array {
   void *data;
   size_t sizeobj;

   public:
   void set(int p, void *v) {
      memcpy(&data[p*sizeobj],
              v,   sizeobj);
   }

   void* get(int p) {
      return &data[p*sizeobj];
   }

   Array(size_t s) : sizeobj(s) {
      data = malloc(64 * sizeobj);
   }
         \end{lstlisting}
      \end{column}
      \begin{column}{.40\linewidth}
         \begin{lstlisting}[style=normalnonumbers]
class Array_int {
   int data[64];


   public:
   void set(int p, int v){
      data[p] = v;
   }


   int get(int p) {
      return data[p];
   }

         \end{lstlisting}
      \end{column}
   \end{columns}
\end{frame}
\note[itemize] {
\item Una alternativa al c\'odigo repetido es usar un \lstinline[style=normal]!void*! y el heap.
\item Tendremos que hacer la copia bit a bit (no se llama a ningun constructor por copia u operador asignaci\'on). Esto puede traer varios problemas para objetos que necesitan copiarse de manera no tan trivial.
\item Con el \lstinline[style=normal]!void*! ganamos generalidad, pero nos arriesgamos a castear manzanas con bananas.
\item Tenemos que saber cual es el tama\~no del objeto.
}

\begin{frame}[fragile]{void* nightmare}
   \begin{columns}[t]
      \begin{column}{.55\linewidth}
         \begin{lstlisting}[style=normal,linebackgroundcolor={%
         \only<1>{\def\lst@linebgrdcmd####1####2####3{}}%
         \btLstHLB<2>{6}% uso de literal
         \btLstHLB<3>{8}% copia
         \btLstHLB<4>{8}% casteo
   }]
// Array version void* (enjoy!)

Array my_ints(sizeof(int));

int i = 5;
my_ints.set(0, &i);

int j = *(int*)my_ints.get(0);
         \end{lstlisting}
      \end{column}
      \begin{column}{.40\linewidth}
         \begin{lstlisting}[style=normalnonumbers,linebackgroundcolor={%
         \only<1>{\def\lst@linebgrdcmd####1####2####3{}}%
         \btLstHLB<2>{6}% uso de literal
         \btLstHLB<3-4>{8}% copia y casteo
   }]
// Array original

Array_int my_ints;


my_ints.set(0, 5);

int j = my_ints.get(0);
         \end{lstlisting}
      \end{column}
   \end{columns}
\vphantom{X}
La implementaci\'on con \lstinline[style=normal]!void*! es gen\'erica pero...\\
\only<2>{no podemos usar literales;}
\only<3>{tenemos que castear!}
\only<4>{tenemos que dereferencear;}
\end{frame}
\note[itemize] {
\item No podemos guardar un literal \lstinline[style=normal]!set(0, 1)!; tenemos que usar una variable \lstinline[style=normal]!set(0, \&i)!
\item La copia del objeto retornado es hecha en el caller
\item Es trivial cometer un error de casteo!
}

\begin{frame}[plain,fragile]{Alternativa II: Precompilador m\'agico}
   \begin{columns}[t]
      \begin{column}{.52\linewidth}
         \begin{lstlisting}[style=normal]
#define MAKE_ARRAY_CLASS(TYPE)\\
  class Array_##TYPE         \\
     TYPE data[64];          \\
                             \\
     public:                 \\
     void set(int p, TYPE v){\\
        data[p] = v;         \\
     }                       \\
                             \\
     TYPE get(int p) {       \\
        return data[p];      \\
     }                       \\
  }//<- fin de la macro sin ;
         \end{lstlisting}
      \end{column}
      \begin{column}{.42\linewidth}
         \begin{lstlisting}[style=normalnonumbers]

class Array_int {
   int data[64];

   public:
   void set(int p, int v) {
      data[p] = v;
   }

   int get(int p) {
      return data[p];
   }
}; // aca si incluyo un ; !!
         \end{lstlisting}
      \end{column}
   \end{columns}
   \begin{lstlisting}[style=normal]
MAKE_ARRAY_CLASS(int);  // instanciacion de las clases
MAKE_ARRAY_CLASS(char); // Array_int y Array_char
   \end{lstlisting}
\end{frame}
\note[itemize] {
\item La idea es crear una macro para crear m\'ultiples clases parecidas. No esta mal, pero es muy d\'ificil de debugear.
\item Requiere la instanciaci\'on expl\'icita de las clases y es f\'acil que alguien instancie dos veces la misma clase (llame a la macro dos veces con los mismos argumentos).
\item Observaci\'on: cuando se haga la macro, no poner el \lstinline[style=normal]!;! al final de esta!
}

\subsection{Templates}
\begin{frame}[fragile]{Templates: un \'unico c\'odigo para gobernarlos a todos}
   \begin{columns}[t]
      \begin{column}{.5\linewidth}
         \begin{lstlisting}[style=normal,linebackgroundcolor={%
         \only<1>{\def\lst@linebgrdcmd####1####2####3{}}%
         \btLstHLB<2>{1}% template keyword
         \btLstHLB<3>{3,6,10}% uso de T
   }]
template<class T>
class Array {
   T data[64];

   public:
   void set(int p, T v) {
      data[p] = v;
   }

   T get(int p) {
      return data[p];
   }
};
         \end{lstlisting}
      \end{column}
      \begin{column}{.42\linewidth}
         \begin{lstlisting}[style=normalnonumbers,linebackgroundcolor={%
         \only<1-2>{\def\lst@linebgrdcmd####1####2####3{}}%
         \btLstHLB<3>{3,6,10}% uso de T
   }]

class Array_int {
   int data[64];

   public:
   void set(int p, int v) {
      data[p] = v;
   }

   int get(int p) {
      return data[p];
   }
};
         \end{lstlisting}
      \end{column}
   \end{columns}
\end{frame}
\note[itemize] {
\item La idea es similar a la alternativa del precompilador: usar el mismo c\'odigo
     pero reemplazando el tipo particular \lstinline[style=normal]!int/char! por uno gen\'erico \lstinline[style=normal]!T!.
\item Pero a diferencia del precompilador, el c\'odigo template es procesado por
     el compilador: es m\'as seguro, hay chequeo de tipos y los errores estan
     mejor explicados (bueno, hasta cierto punto)
}

\begin{frame}[fragile]{Templates: un \'unico c\'odigo para gobernarlos a todos}
   \begin{columns}
      \begin{column}{.5\linewidth}
         \begin{lstlisting}[style=normal]
Array<int> my_ints;

my_ints.set(0, 5);
int j = my_ints.get(0);
         \end{lstlisting}
      \end{column}
      \begin{column}{.42\linewidth}
         \begin{lstlisting}[style=normalnonumbers]
Array_int my_ints;

my_ints.set(0, 5);
int j = my_ints.get(0);
         \end{lstlisting}
      \end{column}
   \end{columns}
\vphantom{X}
Usamos \lstinline[style=normal]!Array<int>! para instanciar el array
y la clase si no fue ya instanciada.
\end{frame}

\begin{frame}[fragile]{Programaci\'on gen\'erica}
   \begin{columns}
      \begin{column}{.58\linewidth}
         \begin{lstlisting}[style=normalnonumbers,linebackgroundcolor={%
         \only<1,4>{\def\lst@linebgrdcmd####1####2####3{}}%
         \btLstHLB<2>{1,3-4}% uso de T y U
         \btLstHLB<3>{7,9}% uso de int y default
   }]
template<class T, class U>
struct Dupla {
   T first;
   U second;
};

template<class T=char, int size=64>
class Array {
   T data[size];
};

Array<> a; // T = char, size = 64
Array<int, 32> b;

         \end{lstlisting}
      \end{column}
      \begin{column}{.4\linewidth}
         \begin{lstlisting}[style=normalnonumbers,linebackgroundcolor={%
         \only<1-3>{\def\lst@linebgrdcmd####1####2####3{}}%
         \btLstHLB<4>{1,2}% uso funciones templates
   }]
template<class T>
void swap(T &a, T &b) {
   T tmp = a;
   a = b;
   b = tmp;
}
         \end{lstlisting}
         \begin{itemize}
            \only<1> {
            \item Containers y algoritmos
                }
            \only<2> {
            \item M\'ultiples par\'ametros
                }
            \only<3> {
                \item Valores por default
                }
            \only<4> {
                \item Funciones templates
                }
         \end{itemize}
      \end{column}
   \end{columns}
\end{frame}
\note[itemize] {
\item No solo las clases pueden ser templates, los \lstinline[style=normal]!struct! y las funciones tambi\'en.
\item Se puede parametrizar por tipo (\lstinline[style=normal]!class T!) o por una constante (\lstinline[style=normal]!int size!).
\item Como todo par\'ametro, pueden tener un default.
}


\begin{frame}[fragile]{Deducci\'on autom\'atica de tipos}
         \begin{lstlisting}[style=normal]
template<class T>
void foo(T i) {
    // ...
}


foo<int>(1);  // T = int (explicit)
foo(2);       // T = int (automatic)

foo<char>(3); // T = char (explicit)
         \end{lstlisting}
\end{frame}
\note[itemize] {
\item Al llamar a una funci\'on template podemos especificar sobre que tipos estamos trabajando o podemos dejar que el compilador lo deduzca autom\'aticamente basandose en los par\'ametros.
\item Tener en cuenta que la deducci\'on autom\'atica puede no tener el efecto que uno quiere pues no siempre es f\'acil determinar el tipo de los par\'ametros: el n\'umero \lstinline[style=normal]!3! es un \lstinline[style=normal]!int! o un \lstinline[style=normal]!char!?
}

\begin{frame}[plain,fragile]{Optimizaci\'on por tipo - Especializaci\'on de templates}
   \begin{lstlisting}[style=normal,linebackgroundcolor={%
         \only<1>{\def\lst@linebgrdcmd####1####2####3{}}%
         \btLstHLB<2>{1-2}% template original
         \btLstHLB<3>{4-5,9,16}% template especializado
         %\btLstHLB<3>{9,16}% misma interfaz
   }]
template<class T>          // Template
class Array { /*...*/ };   // anterior

template<>
class Array<bool> {
   char data[64/8];

   public:
   void set(int p, bool v) {
      if (v)
         data[p/8] = data[p/8] | (1 << (p%8));
      else
         data[p/8] = data[p/8] & ~(1 << (p%8));
   }

   bool get(int p) {
      return (data[p/8] & (1 << (p%8))) != 0;
   }
   \end{lstlisting}
\end{frame}
\note[itemize] {
\item Se permite definir una implementaci\'on espec\'ifica para un tipo en especial.
\item La especializaci\'on de templates es usado en casos de optimizaci\'on o algun otro tipo de customizaci\'on
\item Cuando se instancie un array de tipo \lstinline[style=normal]!Array<bool!>, se utilizar\'a la implementaci\'on optimizada, mientras que
     el template \lstinline[style=normal]!Array<T>! gen\'erico se usara en el resto de los casos. El compilador siempre elegir\'a la especializaci\'on mas espec\'ifica.
\item La versi\'on especializada debe definir los mismo m\'etodos que su par gen\'erico,
     pero la implementaci\'on puede ser completamente distinta.
}

\begin{frame}[fragile]{Polimorfismo en tiempo de compilaci\'on}{}
   \begin{lstlisting}[style=normal,linebackgroundcolor={%
         \only<1,3>{\def\lst@linebgrdcmd####1####2####3{}}%
         \btLstHLB<2>{3}% a==b para strings
   }]
template<class T>
bool cmp(T &a, T &b) {
   return a == b;
}
   \end{lstlisting}
   \pause
La especializaci\'on no solo sirve para optimizar sino para hacer c\'odigo mas razonable.
   \begin{lstlisting}[style=normal,firstnumber=5,linebackgroundcolor={%
         \only<1,3>{\def\lst@linebgrdcmd####1####2####3{}}%
         \btLstHLB<2>{7}% a==b para strings
   }]
template<>
bool cmp<const char*>(const char* &a, const char* &b) {
   return strncmp(a, b, MAX);
}
   \end{lstlisting}
   \pause
   \begin{lstlisting}[style=normal,firstnumber=9]
cmp(1, 2);
cmp("hola", "mundo");
   \end{lstlisting}
\end{frame}
\note[itemize] {
\item No tiene mucho sentido comparar dos punteros a \lstinline[style=normal]!char!, tal vez tiene m\'as 
     sentido hacer una comparaci\'on de strings.
\item Es como una especie de polimorfismo en tiempo de compilaci\'on pues el c\'odigo que se ejecuta depende del tipo de sus argumentos aunque esta decisi\'on se toma en tiempo de compilaci\'on.
}


\subsection{Internals}
\begin{frame}[fragile]{Detras de la magia}
Veamos las implicaciones de este c\'odigo:
   \begin{lstlisting}[style=normal]
Array<int> my_ints;
my_ints.get(0);

Array<int> other_ints;
other_ints.set(0,1)
   \end{lstlisting}
\end{frame}


\begin{frame}[fragile]{Detras de la magia: generaci\'on de c\'odigo m\'inimo}
   \begin{lstlisting}[style=normal]
Array<int> my_ints;
   \end{lstlisting}
   \begin{itemize}
      \item No existe la \alert{clase} \lstinline[style=normal]!Array<int>!
         \begin{itemize}
            \item Se busca \dots
            \begin{itemize}
               \item un template especializado \lstinline[style=normal]!Array<T>! con \lstinline[style=normal]!T = int! (no hay)
               \item un template parcialmente especializado (no hay)
               \item un template gen\'erico \lstinline[style=normal]!Array<T>! (encontrado!)
            \end{itemize}
            \item Se \alert{instancia} la clase \lstinline[style=normal]!Array<int>!
            \item Se crea solo c\'odigo para el constructor y destructor.
         \end{itemize}
      \item Se crea c\'odigo para llamar al constructor e instanciaciar el \alert{objeto} \lstinline[style=normal]!my_ints!
   \end{itemize}
   \pause
   \begin{lstlisting}[style=normal,firstnumber=2]
my_ints.get(0);
   \end{lstlisting}
   \begin{itemize}
      \item No est\'a creado el c\'odigo para el m\'etodo \lstinline[style=normal]!Array<int>::get!, se lo crea y compila.
      \item Se crea c\'odigo para llamar al m\'etodo.
   \end{itemize}
\end{frame}
\note[itemize] {
\item C\'odigo template no usado es c\'odigo no compilado: es muy f\'acil creer que algo esta bien codeado y darnos cuenta al momento de usarlo que no lo est\'a.
\item C++ solo generara c\'odigo desde un template si lo necesita y si solo no existe previamente.
}

\begin{frame}[fragile]{Detras de la magia: generaci\'on de c\'odigo m\'inimo}
   \begin{lstlisting}[style=normal,firstnumber=4]
Array<int> other_ints;
   \end{lstlisting}
   \begin{itemize}
      \item Ya existe la \alert{clase} \lstinline[style=normal]!Array<int>!
      \item Directamente se crea c\'odigo para llamar al constructor.
   \end{itemize}
   \begin{lstlisting}[style=normal,firstnumber=5]
other_ints.set(0, 1);
   \end{lstlisting}
   \begin{itemize}
      \item No est\'a creado el c\'odigo para el m\'etodo \lstinline[style=normal]!Array<int>::set!, se lo crea y compila.
      \item Se crea c\'odigo para llamar al m\'etodo.
   \end{itemize}
\end{frame}


\begin{frame}[fragile]{Copy Paste Programming autom\'atico}
   \begin{lstlisting}[style=normal]
template<class T>
class Array { /*...*/ };

class A { /*...*/ };
class B: public A { /*...*/ };
class C { /*...*/ };

Array<A*> a;
Array<B*> b;
Array<C*> c;
Array<A>  d;
Array<B>  e;
Array<A>  f;
   \end{lstlisting}
Cu\'antas clases \lstinline[style=normal]!Array!s se construyeron?
\uncover<2>{\alert{5!} Un \lstinline[style=normal]!Array! para \lstinline[style=normal]!A*!, otro para \lstinline[style=normal]!B*!, ... Hay c\'odigo copiado y pegado 5 veces (code bloat).}
\end{frame}
\note[itemize] {
\item C++ simplemente toma el template y de \'el genera c\'odigo al mejor estilo copy and paste
\item Como los tipos \lstinline[style=normal]!A*!, \lstinline[style=normal]!B*! y \lstinline[style=normal]!C*! son tipos distintos, C++ generara 3 clases una para cada uno de esos tipos usando como \lstinline[style=normal]!Array<T>! como template: esto termina en un ejecutable mucho mas grande de lo necesario (code bloat).
}

\begin{frame}[fragile]{Especializaci\'on parcial de templates}
   \begin{lstlisting}[style=normal,linebackgroundcolor={%
         \only<1>{\def\lst@linebgrdcmd####1####2####3{}}%
         \btLstHLB<2>{1-2}% template generico T
         \btLstHLB<3>{4-5}% template especializado para void*
         \btLstHLB<4>{7-8}% template parcialmente especializado para T*
         \btLstHLB<5>{11,15}% solo casteos, toda la implementacion queda del lado de Array<void*>
   }]
template<class T> // Template generico
class Array { /*...*/ };

template<>        // Especializacion completa para void*
class Array<void*> { /*...*/ };

template<class T> // Especializacion parcial para T*
class Array<T*> : private Array<void*> {
   public:
   void set(int p, T* v) {
      Array<void*>::set(p, v);
   }

   T* get(int p) {
      return (T*) Array<void*>::get(p);
   }
};
   \end{lstlisting}
\end{frame}
\note[itemize] {
\item El problema de la implementaci\'on original que usaba \lstinline[style=normal]!void*!
     eran los peligrosos casteos y la p\'erdida de chequeos de tipos.
\item   Una especializaci\'on parcial nos permite encapsular los casteos en
     un template, liberando al usuario de ellos mientras que la mayoria
     de la implementaci\'on del container esta contenida en el \lstinline[style=normal]!Array<void*>!
}

\begin{frame}[fragile]{Especializaci\'on parcial de templates}
   \begin{lstlisting}[style=normal,firstnumber=8]
Array<A*> a;
Array<B*> b;
Array<C*> c;
Array<A>  d;
Array<B>  e;
Array<A>  f;
   \end{lstlisting}
Y ahora, cu\'antas clases \lstinline[style=normal]!Array!s se construyeron?
   \uncover<2->{\alert{6!}}
   \begin{itemize}
      \item<2-> 2 clases usando \lstinline[style=normal]!Array<T>! con \lstinline[style=normal]!T = A! y \lstinline[style=normal]!T = B!
      \item<2-> 3 clases usando \lstinline[style=normal]!Array<T*>! con \lstinline[style=normal]!T = A!, \lstinline[style=normal]!T = B! y \lstinline[style=normal]!T = C!
      \item<2-> 1 clase m\'as para \lstinline[style=normal]!Array<void*>!
   \end{itemize}
   \uncover<2->{M\'as clases, es peor!?}
   \uncover<3->{Como \lstinline[style=normal]!Array<T*>! son puros casteos, el compilador se encargar\'a de hacer inline y remover el c\'odigo superfluo. M\'as compacto y m\'as r\'apido.}
\end{frame}
\note[itemize] {
\item Para los tipos \lstinline[style=normal]!A*!, \lstinline[style=normal]!B*! y \lstinline[style=normal]!C*! se crearan 3 clases con \lstinline[style=normal]!Array<T*>! como su template y como este hereda de \lstinline[style=normal]!Array<void*>! tambien se creara esa clase dandonos un total de 4 clases. Para los tipos A y B se crearan dos clases adicionales usando \lstinline[style=normal]!Array<T>! como template dando un conteo total de 6 clases.
\item Pero observando en detalle, \lstinline[style=normal]!Array<T*>! solo tiene c\'odigo de casteo (seguro) y m\'etodos
     de una sola l\'inea. Las clases instanciadas para los tipo \lstinline[style=normal]!A*!, \lstinline[style=normal]!B*! y \lstinline[style=normal]!C*! no
     supone un aumento considerable del c\'odigo (evitamos el code bloat).
\item Mas aun, con m\'etodos de una sola l\'inea, puede que el compilador los
     haga inline, optimizando el tama\~no del ejecutable y el tiempo de ejecuci\'on.
}


\begin{frame}[fragile]{Resumen - Templates}
   \begin{itemize}
      \item \alert{Jam\'as} implementar un template \uncover<2->{al primer intento. Crear una clase prototipo (\lstinline[style=normal]!Array_ints!, \alert{testearla} y luego pasarla a template \lstinline[style=normal]!Array<T>!}
      \item<3-> Si se va a usar el templates con punteros, evitar el code bloat implementando la especializaci\'on \lstinline[style=normal]!void*! (\lstinline[style=normal]!Array<void*>!) y luego la especializaci\'on parcial \lstinline[style=normal]!T*! (\lstinline[style=normal]!Array<T*>!).
      \item<4-> Opcionalmente, implementar especializaciones optimizadas (\lstinline[style=normal]!Array<bool>!)
   \end{itemize}
\end{frame}


\appendix
\section<presentation>*{\appendixname}
\subsection<presentation>*{Referencias}

\begin{frame}[allowframebreaks]
   \frametitle<presentation>{Referencias}

   \begin{thebibliography}{10}

         \beamertemplatebookbibitems
         % Start with overview books.
      \bibitem{cplusplus}
         http://cplusplus.com

      \bibitem{Sutter}
         Herb Sutter.
         \newblock {\em Exceptional C++: 47 Engineering Puzzles}.
         \newblock Addison Wesley, 1999.

      \bibitem{Stroustrup}
         Bjarne Stroustrup.
         \newblock {\em The C++ Programming Language}.
         \newblock Addison Wesley, Fourth Edition.

         %\beamertemplatearticlebibitems
         % Followed by interesting articles. Keep the list short.

   \end{thebibliography}
\end{frame}

\end{document}


